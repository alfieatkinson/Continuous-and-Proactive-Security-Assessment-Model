\section{Introduction of the Continuous and Proactive Security Assessment Model (CAPSAM)}
Introduce CAPSAM as a "top-down" approach to information security risk assessment. State that this model addresses the limitations of reactive approaches and the ABV. Emphasise that it is designed to integrate security into every stage of system development.

\section{Theoretical Foundations of CAPSAM}
\subsection{DevSecOps Principles}
Explain that DevSecOps integrates security into all phases of the software development lifecycle. Emphasise the `shift left' principle which promotes security from the beginning of development. Show how CAPSAM aligns with these principles by integrating security from the earliest stage. \citep{ibm2021devsecops}
\subsection{Risk Management Theories}
Contrast CAPSAM with Financial Theory, Agency Theory, Stakeholder Theory, and New Institutional Economics. Point out that, while relevant, these theories do not provide a framework for continuous and proactive security. Highlight that the theories from the MPRA paper have low empirical verification. Use the New Institutional Economics to justify the consideration of governance processes and socio-economic institutions. \citep{klimczak2007risk}

\section{Components of CAPSAM}
\subsection{Initial Risk Assessment}
Explain the need for a comprehensive initial assessment at multiple levels (system, component, feature). Highlight the importance of worst-case scenario planning.
\subsection{Proactive Measures}
Describe the necessity of integrating security by design. Emphasise the need for top management involvement. Explain the importance of regular employee training.
\subsection{Continuous Risk Assessment}
Explain the need for ongoing assessments throughout the system's lifecycle. Highlight the importance of feature-level assessments. Describe the need to monitor the evolving threat landscape.
\subsection{Incident Response Planning}
Explain the need for predefined incident response plans. Highlight the importance of taking immediate and decisive action.
\subsection{Regular Audits and Reviews}
Explain the need for regular internal audits. Highlight the value of external reviews.
\subsection{Feedback Loop}
Describe the need to use findings from assessments and incidents to improve the system. Highlight how this iterative process ensures that the system adapts to new threats.

\section{Importance of TMT Buy-In for CAPSAM}
Argue that TMT involvement is crucial for aligning security with business objectives. Explain that TMT engagement fosters a proactive security culture.

\section{Benefits of CAPSAM Over Reactive Approaches}
Highlight that it promotes a proactive rather than reactive security stance. Emphasise that CAPSAM encourages continuous vigilance. State that it offers a holistic approach to security. Demonstrate that CAPSAM is adaptable to new threats.

\section{Real-Life Examples}
Include case studies or real-life examples to back up your points. Reference literature that supports proactive security. Discuss how known data breaches could have been prevented by proactive models like CAPSAM.

\section{CAPSAM as a Strategic Approach}
Show that the model treats security risk as a business issue, not just an IT concern. Explain how TMT involvement in CAPSAM aligns security with strategic objectives.

\section{Implementation Stages of CAPSAM}
Describe the four stages: initiation, design and development, operational, and feedback and improvement.