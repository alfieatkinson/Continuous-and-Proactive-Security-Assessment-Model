\section{Introduction to the New Model}
This chapter introduces a new ``top-down'' Continuous and Proactive Security Assessment Model (CAPSAM) framework as a response to limitations in traditional risk assessment approaches. The case study by \citet{shaikh2023information} addressed the reactive nature of the Top Management Team (TMT)'s attention in its influence on the decision to carry out an Information Security Risk Assessment (ISRA). This reveals the need for a new proactive, continuous security model that integrates information security across all layers of an organisation.

\section{Overview of the CAPSAM Framework}
    \subsection{Purpose, Goals, and Intended Outcomes}
    The CAPSAM framework is designed to \textbf{address the limitations of cybersecurity models} by emphasising a proactive and continuous approach to risk assessment. Its primary purpose is to \textbf{integrate information security considerations from the earliest stages of system development} and throughout its lifecycle.

    The main goal of CAPSAM is to minimise the likelihood and impact of cybersecurity breaches through vigilant, ongoing risk management. The model aims to \textbf{integrate information security across all layers of an organisation} and focuses on continuous improvement, ensuring a resilient security posture that adapts to the ever-changing threat landscape. By doing this, CAPSAM prioritises the protection of all stakeholders\textemdash including the customer and their data\textemdash strengthening overall organisational security and trust.

    \subsection{The Five Pillars of CAPSAM}
    The core philosophies of CAPSAM can be summarised in five pillars: \textbf{Culture, Continuous, Auditing, Response, Proactive (CCARP)}. These pillars are illustrated in Figure \ref{fig:CAPSAM_Pillars} and form the foundation of the model's approach to information security risk management.

    \begin{figure}[htbp]
        \centering
        \includegraphics[width=0.8\textwidth]{figures/CAPSAM-Pillars.png}
        \caption{The Five Pillars of CAPSAM: Culture, Continuous, Auditing, Response, and Proactive (CCARP) and their components, explained in Table \ref{tab:CAPSAM_Pillars_Components} in appendices.}
        \label{fig:CAPSAM_Pillars}
    \end{figure}

    \subsection{FAMRM Cycle}
    The CAPSAM framework operates within the FAMRM cycle (New \textbf{Feature}, Information Security Risk \textbf{Assessment}, Proactive \textbf{Mitigation} Strategies, Incident \textbf{Response} Planning, and Continuous Threat \textbf{Monitoring}). This cycle ensures that each new feature or system development is subject to proactive mitigation strategies, and continuous risk assessments where feedback loops inform the next ISRA. The FAMRM cycle is illustrated in Figure \ref{fig:FAMRM_Cycle}.

    \begin{figure}[htbp]
        \centering
        \includegraphics[width=0.6\textwidth]{figures/FAMRM-Cycle.png}
        \caption{FAMRM Cycle for Integrating Security into New Feature Development, Highlighting Continuous Risk Assessment, Proactive Mitigation, and Feedback Loops in Agile and DevSecOps Environments.}
        \label{fig:FAMRM_Cycle}
    \end{figure}

\section{Justification for a Top-Down Approach}
    \subsection{Establishing a Strong Security Posture through Executive Leadership}
    A top-down approach to information security risk management ensures strategic alignment and organisational commitment to security. By prioritising executive leadership, strategic priorities are set, resources allocated, and security policies enforced, framing information security as a core corporate governance issue rather than merely a technical concern \citep{linkov2014risk, fazlida2015information}. This alignment mirrors CAPSAM's emphasis on integrating security into organisational culture and fostering a ``security-first mindset'' among all stakeholders, as outlined in the `Culture' pillar.
    
    In contrast, bottom-up approaches focus on identifying technical vulnerabilities but often lack strategic direction, executive support, and adequate resource allocation \citep{fazlida2015information}. Without such oversight, efforts may neglect broader risks, including human and social factors, hindering a cohesive and effective security strategy \citep{shedden2010information}.
    
    \subsection{Corporate Governance and Strategic Integration}
    A top-down approach aligns with corporate governance by ensuring the Board of Directors (BOD) and executive management recognise their responsibility to safeguard information assets. \citet{fazlida2015information} explain that executive involvement integrates security into organisational strategies, enhancing competitive advantage, client satisfaction, and trust. This perspective aligns with CAPSAM's goal of embedding security considerations at all organisational levels, enabling seamless integration of security measures into daily operations. CAPSAM's FAMRM cycle, along with the `Culture' pillar, reinforce this integration by ensuring top management's active engagement and resource allocation.
    
    \subsection{Addressing the Limitations of Bottom-Up Approaches}
    Bottom-up approaches face significant challenges, including limited management buy-in and insufficient coordination across departments \citep{shaikh2023information}. This results in an overemphasis on technical controls while neglecting non-technical factors, such as human error and social engineering risks \citep{shedden2010information}. Furthermore, these strategies often fail to address the complex interactions of technical, social, and economic factors shaping an organisation's risk profile \citep{cai2017cybersecurity}. CAPSAM's top-down emphasis mitigates these issues by aligning policies with organisational needs and fostering a culture of security awareness through its `Culture' and `Proactive' pillars, ensuring holistic risk management.
    
\section{Theoretical Foundations}    
    \subsection{Attention-Based View (ABV) and CAPSAM's Culture Pillar}
    The Attention-Based View (ABV) theory highlights the importance of prioritising issues that are contextually relevant and salient \citep{shaikh2023information}. CAPSAM operationalises this theory by maintaining continuous focus on cybersecurity, thereby ensuring top management prioritises and allocates resources for proactive security measures. The `Culture' pillar reinforces this by securing top management involvement and embedding security as a core organisational value. This approach aligns with findings that management attention significantly increases the likelihood of conducting robust Information Security Risk Assessments (ISRAs) \citep{shaikh2023information}.
    
    \subsection{Agile and DevSecOps Principles in the Continuous Pillar}
    Agile methodologies and DevSecOps principles form a foundation for CAPSAM's `Continuous' pillar, promoting adaptability, speed, and integration of security into the development lifecycle \citep{ibm2021devsecops, dingsoyr2012agile}. Agile's iterative approach enables rapid adjustments to evolving threats, while DevSecOps emphasises visibility and auditability. The FAMRM cycle embodies these principles by incorporating ongoing risk assessments and iterative feedback loops, aligning security measures with the dynamic threat landscape \citep{ibm2021devsecops}.
    
    \subsection{ISO 31000: Risk Management Standard and the FAMRM Cycle}
    ISO 31000 provides a comprehensive framework for risk management, emphasising communication, monitoring, and continuous improvement \citep{purdy2010iso}. CAPSAM integrates these principles through the FAMRM cycle, ensuring a proactive and iterative approach to risk management. By focusing on dynamic assessments and mitigation strategies, CAPSAM addresses the limitations of static risk models, aligning with ISO 31000's definition of risk as the ``effect of uncertainty on objectives'' \citep{purdy2010iso}.
    
    \subsection{Organisational Learning and Feedback in CAPSAM}
    CAPSAM emphasises continuous learning through feedback loops, refining security measures based on insights from incident responses, audits, and risk assessments. This iterative approach, central to the FAMRM cycle, aligns with organisational learning principles advocating adaptation and sustained improvement \citep{murray2003continuous}. Regular audits and third-party reviews further support this dynamic model, enhancing resilience against evolving threats.
    
    \subsection{Trust and Customer-Focused Security}
    CAPSAM prioritises customer-focused security by embedding data protection at every stage of system development, reflecting stakeholder theory and corporate social responsibility (CSR) principles \citep{moir2001csr, parmar2010stakeholder}. The Culture pillar's emphasis on transparency and proactive incident response aligns with trust theory, which underscores the role of organisational credibility in fostering stakeholder trust \citep{castelfranchi2010trust}. Trust is treated as relational capital, benefiting both the organisation and its stakeholders by enhancing credibility and reliability \citep{castelfranchi2010trust}.      

\section{Critical Analysis and Justification}
    \subsection{Strengths of CAPSAM}
    Critically analyse the strengths of the CAPSAM model, such as its proactive nature, emphasis on stakeholder engagement, and its continuous improvement cycle. Use sources to support your claims on the benefits of proactive risk management.

    \subsection{Limitations of CAPSAM}
    Acknowledge the limitations and challenges of implementing CAPSAM. Discuss aspects such as the need for continuous updates to risk assessments, the resources required for ongoing monitoring, and the potential for over-reliance on specific teams or individuals.

    \subsection{Comparison with Case Study Approach}
    Compare CAPSAM with the reactive approach in the case study. Highlight how CAPSAM's proactive, top-down, and continuous nature addresses the gaps and limitations of the case study approach.

    \subsection{Challenges in Implementing CAPSAM}
    Discuss potential challenges in implementing CAPSAM, such as obtaining management support, ensuring consistent communication across departments, and managing the resources needed for continuous monitoring.

\section{Implementation of CAPSAM}
    \subsection{Phase 1: Planning and Preparation}
    Outline the first phase of implementing CAPSAM, which involves identifying key stakeholders, defining roles and responsibilities, and establishing communication channels. Discuss the importance of aligning the security risk management policy with the organisation's overall strategy.

    \subsection{Phase 2: Implementation}
    Describe the process of conducting risk assessments, identifying vulnerabilities, prioritizing risks, and developing mitigation strategies. Explain the importance of integrating risk management activities with the software development lifecycle.

    \subsection{Phase 3: Review and Improvement}
    Detail the review phase, which involves regular assessments of the CAPSAM framework, adapting the model based on insights and business changes, and fostering a culture of continuous security improvement.

\section{Conclusion}
Summarize the key points discussed in the chapter, reinforcing the value of the CAPSAM framework in overcoming the limitations of traditional risk assessment models. Emphasize the importance of a proactive, continuous approach to information security and its alignment with business practices and strategic goals.