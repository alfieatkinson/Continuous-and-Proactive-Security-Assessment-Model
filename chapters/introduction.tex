\section{Background on Cybersecurity Risks}
Cybersecurity focuses on protecting digital devices, networks, and information from unauthorized access and data breaches \citep{mijwil2023exploring}. It employs techniques such as firewalls, encryption, secure passwords, and threat detection. Cybersecurity risk is a product of vulnerabilities, threats, and potential impacts of cyber-attacks. Vulnerabilities are system weaknesses, while threats include attacks that exploit these weaknesses \citep{prasad2020cyber}. The Internet and IoT are major threat sources, with phishing attacks becoming more sophisticated, rendering passwords insufficient for security. Raising awareness is crucial for managing digital environments and defending against electronic threats \citep{mijwil2023exploring}.

\section{The Role of Information Security Risk Assessments (ISRAs)}
Information Security Risk Assessments (ISRAs) identify and manage vulnerabilities by analysing critical information assets and guiding mitigation strategies \citep{shedden2010information}. They help organisations prioritise security efforts, focusing on key assets and vulnerabilities, and determining cost-effective risk reduction strategies \citep{shedden2010information}.

\section{Top Management Team (TMT) Involvement}
TMT involvement is essential for effective risk management, ensuring that risk identification and control systems are properly implemented \citep{fazlida2015information}. Engaging TMT integrates cybersecurity into the organisation's strategy, securing necessary resources and attention, and ensuring compliance with legal obligations \citep{shaikh2023information}.

\section{Purpose and Structure}
This report evaluates \citet{shaikh2023information}'s Attention-Based View (ABV) Theory, examining its strengths and limitations, and introduces a new Continuous and Proactive Security Assessment Model (CAPSAM). CAPSAM offers a proactive approach to cybersecurity, addressing ABV's reactive limitations, with the goal of presenting CAPSAM as an alternative for more effective security planning. The report is structured into four main sections: This introduction, an evaluation of the ABV, the introduction, justification, and critical analysis of CAPSAM, a conclusion of the report. The body of this report is 3,290 words long\footnote{Word count excludes headings, captions, references, and appendices.}.