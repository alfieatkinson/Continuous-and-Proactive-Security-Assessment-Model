\section{Background on Cybersecurity Risks}
Cybersecurity is a rapidly growing field focused on safeguarding digital devices, networks, and information from unauthorised access and preventing data theft or alteration \citep{mijwil2023exploring}. It employs a range of techniques, processes, and practices to protect sensitive information and deter cyber attacks. Tactics for protecting against cyber attacks include firewalls, encryption, secure passwords, and threat detection and response systems, and employees should be trained on these strategies.

Cybersecurity risk is determined by the combination of vulnerabilities, threats, and the potential impact of cyber-attacks. Vulnerabilities are the weaknesses present in the system, and threats are the possibilities of cyber-attacks that exploit these vulnerabilities \citep{prasad2020cyber}. The Internet and the Internet of Things (IoT) are significant sources of threats, while phishing attacks are becoming increasingly sophisticated, and passwords alone are no longer sufficient for ensuring security. Raising awareness about cybersecurity risks is imperative for effectively handling digital environments and safeguarding them against electronic threats \citep{mijwil2023exploring}.

\section{The Role of Information Security Risk Assessments (ISRAs)}
Information Security Risk Assessments (ISRAs) are a key tool for identifying and managing vulnerabilities. ISRAs help organisations to identify their security risks and provide a measured analysis of their critical information assets, which informs the development of plans to mitigate these risks \citep{shedden2010information}. These assessments are essential for protecting IT assets and form the basis for a secure information system. \citet{shedden2010information} also say ISRAs enable organisations to prioritise their security efforts, focusing on the most critical assets and vulnerabilities as well as helping organisations determine the most cost-effective way to reduce risks.

\section{Top Management Team (TMT) Involvement}
The Top Management Team (TMT) involvement is vital for effective risk management as they are ultimately responsible for ensuring that due diligence is undertaken in identifying risk and implementing effective systems of controls \citep{fazlida2015information}. TMT engagement can ensure that cybersecurity is viewed as an integral part of the organisation rather than a technical issue handled solely by IT. This involvement ensures a holistic approach to security, with the TMT championing risk assessment exercises and ensuring that cybersecurity receives the necessary resources and attention \citep{shaikh2023information}. \citet{fazlida2015information} also state that TMT attention to security is important to ensure that risk is reduced and the organisation meets its legal obligations.

\section{Purpose of the Report}
This report evaluates the use of the Attention-Based View (ABV) Theory by \citet{shaikh2023information}, exploring its strengths and limitations. The report will then introduce a new Continuous and Proactive Security Assessment Model (CAPSAM) as a solution. This model addresses the need for a proactive approach to cybersecurity, in contrast to the reactive focus of the ABV. The aim of this report is to critically appraise the ABV, revealing its shortcomings in proactive security planning, and then present CAPSAM as a proactive alternative.