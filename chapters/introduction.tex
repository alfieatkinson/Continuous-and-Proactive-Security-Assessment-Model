\section{Background on Cybersecurity Risks}
Cybersecurity is a critical field aimed at protecting digital environments from unauthorized access and data theft \citep{mijwil2023exploring}. The increasing prevalence of cyber threats has led to rapid growth in cybersecurity measures, such as firewalls, encryption, and threat detection systems. Organizations must prioritize employee training on these tactics to enhance security \citep{mijwil2023exploring}. Cyber risk is influenced by vulnerabilities, threats, and potential attack impacts \citep{prasad2020cyber}. Major threat sources include the Internet, IoT, and web applications, with phishing attacks becoming more sophisticated and passwords alone being insufficient for security. The primary threat actors are cyber criminals, nation states, and hacktivists \citep{prasad2020cyber}. Raising awareness about cybersecurity risks is crucial for effectively managing digital environments and protecting against electronic threats \citep{mijwil2023exploring}.

\section{The Role of Information Security Risk Assessments (ISRAs)}
Introduce ISRAs as a key tool for identifying and managing vulnerabilities. Note that ISRAs are essential for protecting IT assets. Explain that risk assessments help with prioritising security efforts.

\section{Top Management Team (TMT) Involvement}
Explain that the TMT has a critical role in cybersecurity governance. Mention that their involvement is vital for effective risk management. Indicate that TMT engagement can ensure a holistic approach to security.

\section{Purpose of the Report}
State that this report evaluates the case study's use of the Attention-Based View (ABV) theory. Introduce your proposed Continuous and Proactive Security Assessment Model (CAPSAM) as a solution. Clearly state the aim of the report, which is to critically appraise the ABV, and present a proactive approach through CAPSAM.