\section{Summary of the Case Study Evaluation}
The case study effectively uses the ABV theory to explain how a significant cybersecurity breach captures the attention of the TMT. However, ABV is a reactive model, focusing on responses after a breach and lacking guidance on proactive measures. The model's emphasis on learning from failures highlights the need for a more comprehensive security governance approach, including prevention.

\section{Key Features of CAPSAM}
CAPSAM integrates security across an organisation through five core components: Culture (security-first mindset), continuous risk assessment, auditing (regular reviews), response plans for incidents, and a proactive approach (worst-case scenario planning). CAPSAM's iterative approach ensures a resilient security posture, offering a clear improvement over reactive models.

\section{Role of the TMT}
The TMT plays a crucial role in CAPSAM's success by championing security as a strategic objective and ensuring proper resource allocation. Unlike models that treat security as a technical issue, CAPSAM requires TMT involvement to create a security-first culture, making their commitment vital for effective implementation.

\section{Benefits of CAPSAM}
CAPSAM enhances cybersecurity by reducing the likelihood of successful cyberattacks through its proactive approach. Its continuous nature enables dynamic responses to emerging threats, ensuring resilience over time. By embedding security into all development stages, CAPSAM offers robust protection for organisational assets and consumer trust, making security a core component of the system and culture.