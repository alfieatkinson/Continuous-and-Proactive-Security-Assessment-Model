This report has evaluated the ABV theory and introduced the CAPSAM framework as a superior alternative for cybersecurity risk management. While the ABV explains why TMT prioritise cybersecurity after a breach due to its focus on negative events, it is inherently reactive. Its core principles\textendash focus of attention, structural distribution of attention, and situated attention\textendash show how breaches drive TMT attention but do not provide a proactive security framework. The ABV's reactive nature means firms continually respond to threats instead of anticipating them, which is unsustainable.

CAPSAM addresses these limitations with a proactive, top-down approach. Built on five pillars\textendash Culture, Continuous, Auditing, Response, and Proactive (CCARP)\textendash it emphasises integrating security from system development's early stages. The model follows the FAMRM cycle, embedding security at every stage. By prioritising executive leadership, CAPSAM aligns security with organisational goals and fosters a security-first mindset. This approach overcomes the limitations of bottom-up strategies, which often lack strategic direction and executive support. Theoretical foundations include the ABV, Agile and DevSecOps principles, ISO 31000 standards, organisational learning, and stakeholder/trust theories.

CAPSAM's benefits include enabling organisations to address threats before they arise, reducing cyberattack risks. The model fosters a security-first mindset across the organisation and integrates security into agile development cycles. CAPSAM's continuous improvement cycle, driven by ongoing assessments and feedback, ensures security remains strong against emerging threats. Its focus on customer data protection enhances trust by embedding security throughout development. Real-world data breaches, such as \textit{Target}, \textit{Yahoo}, \textit{Sony PlayStation}, \textit{TJX}, \textit{Equifax}, and \textit{Uber}, highlight the importance of adopting proactive, transparent security practices.

However, CAPSAM has limitations. Implementing it requires continuous updates to risk assessments, which can be resource-intensive, and ongoing monitoring. The model depends on top management's commitment to cybersecurity, which can be a challenge. Regular audits and third-party reviews, including penetration testing, are also needed but can be time-consuming and require external expertise. Additionally, consistently applying CAPSAM across large organisations may be difficult, particularly in overcoming resistance to change and securing buy-in from BODs who may lack IT expertise. Despite these challenges, CAPSAM offers a robust and comprehensive approach to cybersecurity, addressing the shortcomings of reactive models and providing a framework for long-term resilience.