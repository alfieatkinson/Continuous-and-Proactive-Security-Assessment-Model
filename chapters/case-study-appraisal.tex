\section{Summary of the Case Study}
Briefly describe the case study's central question: how do cybersecurity breach costs and TMT attention to cybersecurity influence the decision to conduct an ISRA. Summarise the key findings of the research, for example, that TMT attention increases with higher breach costs and that TMT attention mediates the decision to carry out an ISRA.

\section{Explanation of the Attention-Based View (ABV) Theory}
\subsection{Focus of Attention}
Discuss how limited attention affects decision-making within an organisation.
\subsection{Structural Distribution of Attention}
Detail how an individual's hierarchical position shapes what they pay attention to.
\subsection{Situated Attention}
Explain how immediate events and situations influence the focus of attention.

\section{Statement of Hypotheses}
Clearly state the four hypotheses tested in the paper:
\begin{enumerate}
    \item Higher breach costs increase the likelihood of carrying out an ISRA.
    \item Higher breach costs increase TMT attention to cybersecurity.
    \item TMT attention to cybersecurity has a positive effect on the decision to carry out an ISRA.
    \item TMT attention to cybersecurity mediates the positive effect of cybersecurity breach costs on the decision to carry out an ISRA.
\end{enumerate}

\section{Critical Appraisal of the ABV}
\subsection{Merits/Strengths}
Discuss the ABV's capacity to explain why TMT attention is heightened following a costly breach. Explain the model's ability to show why firms might not act on security until a crisis emerges.
\subsection{Demerits/Weaknesses}
Point out that the ABV focuses on a reactive response to breaches, and overlooks proactive security planning. Explain that the model assumes that TMT attention is driven primarily by negative events, ignoring other influences. Highlight that the ABV does not provide a framework for proactive security measures.

\section{Transition to New Model}
Briefly indicate that the limitations of the ABV highlight the need for a new, proactive model which you will present in the next chapter.