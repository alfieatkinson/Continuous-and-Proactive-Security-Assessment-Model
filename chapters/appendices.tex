\section*{Appendix A: Glossary}
\addcontentsline{toc}{section}{Appendix A: Glossary}

\renewcommand{\arraystretch}{1.3}
\begin{longtable}{|>{\raggedright}m{4cm}|m{10cm}|}
    \caption{Glossary of Key Terms Relevant to the Attention-Based View, CAPSAM Framework, and Risk Management}
    \label{tab:Glossary} \\
    \hline
    \textbf{Term} & \textbf{Definition} \\ \hline\hline
    \endfirsthead
    \hline
    \textbf{Term} & \textbf{Definition} \\ \hline
    \endhead
    \hline
    \endfoot
    \hline
    \endlastfoot

    \textbf{ABV} & Attention-Based View (ABV): A theory that suggests that a firm's behaviour is shaped by how decision-makers allocate their attention. This theory is built on the idea that human rationality is limited, and decision-makers must focus on specific issues to make effective choices. \\ \hline
    \textbf{Agile} & A project management methodology that emphasises iterative development, flexibility, and collaboration, enabling teams to quickly respond to evolving requirements and challenges, integrated into CAPSAM’s ‘Continuous’ pillar. \\ \hline
    \textbf{Auditing Pillar} & A pillar of CAPSAM involving regular internal audits, third-party reviews, and penetration testing. \\ \hline
    \textbf{Bottom-Up Approach} & A tactical approach where security initiatives are driven by technical teams, which may lead to misalignment with organisational goals if not properly overseen. \\ \hline
    \textbf{CAPSAM} & Continuous and Proactive Security Assessment Model (CAPSAM): A top-down framework designed for proactive and continuous risk assessment, integrating information security from the earliest stages of system development. \\ \hline
    \textbf{Continuous Pillar} & A pillar of CAPSAM promoting regular risk assessments throughout a system's lifecycle, continuous monitoring of the threat landscape, and using feedback loops to refine security measures. \\ \hline
    \textbf{Culture Pillar} & A pillar of CAPSAM focusing on fostering a security-first mindset across the organisation, embedding security into agile development and DevSecOps practices, and prioritising customer data protection. \\ \hline
    \textbf{Cybersecurity Risk} & The combination of vulnerabilities, threats, and the potential impact of cyber-attacks. Vulnerabilities are weaknesses in the system, while threats exploit these vulnerabilities. \\ \hline
    \textbf{DevSecOps} & A software development approach that integrates security practices within the DevOps pipeline, emphasising collaboration between development, security, and operations teams. It is integrated into CAPSAM’s ‘Continuous’ pillar. \\ \hline
    \textbf{FAMRM} & FAMRM Cycle: A cyclical process within CAPSAM designed to integrate security into every stage of a system's development, from new feature creation to ongoing threat monitoring. It stands for New Feature, Information Security Risk Assessment, Proactive Mitigation Strategies, Incident Response Planning, and Continuous Threat Monitoring. \\ \hline
    \textbf{Focus of Attention} & A principle of ABV explaining that individuals prioritise issues based on their perceived importance and relevance within a given context due to limited attention capacity. Negative events, such as high-cost cybersecurity breaches, require attention from the Top Management Team (TMT). \\ \hline
    \textbf{ISO 31000} & A comprehensive framework for risk management, emphasising communication, monitoring, and continuous improvement. \\ \hline
    \textbf{ISRA} & Information Security Risk Assessment (ISRA): A process that identifies and evaluates potential threats and vulnerabilities to information assets, enabling organisations to develop risk mitigation strategies. \\ \hline
    \textbf{Proactive Pillar} & A pillar of CAPSAM involving planning for worst-case scenarios, integrating security into every phase of system development, providing employee training, and identifying vulnerabilities. \\ \hline
    \textbf{Response Pillar} & A pillar of CAPSAM focusing on developing and regularly updating incident response plans, conducting post-incident analysis, and maintaining clear communication with stakeholders. \\ \hline
    \textbf{Structural Distribution of Attention} & A principle of ABV stating that an individual's position within an organisation's hierarchy influences what they pay attention to, such as TMTs focusing on security issues due to their fiduciary duty. \\ \hline
    \textbf{Situated Attention} & A principle of ABV that posits an individual's attention is shaped by the immediate situation, like focusing on urgent cybersecurity breaches causing material damage. \\ \hline
    \textbf{TMT} & Top Management Team (TMT): The group of individuals at the highest level of an organisation responsible for ensuring due diligence in risk identification and control implementation. \\ \hline
    \textbf{Top-Down Approach} & A strategic approach where security initiatives and policies are driven by executive leadership, ensuring alignment with overall organisational goals. \\ \hline
\end{longtable}

\newpage

\section*{Appendix B: CAPSAM Pillar Components}
\addcontentsline{toc}{section}{Appendix B: CAPSAM Pillar Components}

\renewcommand{\arraystretch}{1.3}
\begin{longtable}{|>{\raggedright}m{2.5cm}|>{\raggedright}m{3.5cm}|>{\raggedright\arraybackslash}m{8cm}|}
    \caption{Components of the CAPSAM Pillars as shown in Figure \ref{fig:CAPSAM_Pillars}.}
    \label{tab:CAPSAM_Pillars_Components} \\
    \hline
    \textbf{Pillar} & \textbf{Component} & \textbf{Explanation} \\ \hline\hline
    \endfirsthead
    \hline
    \textbf{Pillar} & \textbf{Component} & \textbf{Explanation} \\ \hline
    \endhead
    \hline
    \endfoot
    \hline
    \endlastfoot

    \textbf{Culture} & Security-First Mindset & Cultivating a shared responsibility for security across the organisation, integrating it as a core value at all levels. This ensures that security is a primary consideration in all activities. \\
    \cline{2-3}
    & Top Management Involvement & Securing commitment from top management to provide necessary resources and attention to cybersecurity, ensuring security is viewed as integral, not just a technical issue. \\
    \cline{2-3}
    & Agile \& DevSecOps Integration & Embedding security into agile development cycles and DevSecOps practices, ensuring that security is integrated throughout the entire development lifecycle with iterative, security-conscious practices. \\
    \cline{2-3}
    & Customer-Focused Security & Prioritising consumer data protection and building trust by embedding security at every stage of system development, ensuring responsibility for security and customer service. \\
    \hline\hline

    \textbf{Continuous} & Regular Risk Assessments & Conducting continuous and dynamic risk assessments that are automated, frequent, and responsive to system changes, allowing quick adaptation to emerging risks. \\
    \cline{2-3}
    & Dynamic Threat Landscape & Continuously monitoring and adapting to the evolving threat landscape, ensuring security strategies remain resilient to new risks and attack vectors. \\
    \cline{2-3}
    & Feedback Loop & Using insights from risk assessments, incident responses, and audits to inform and refine future security strategies, promoting continuous improvement. \\
    \hline\hline

    \textbf{Auditing} & Regular Internal Audits & Conducting regular audits to ensure compliance with security policies, assess effectiveness, and uncover areas for improvement, providing internal teams with an opportunity to assess and refine security measures. \\
    \cline{2-3}
    & Third-Party Reviews & Engaging external experts for unbiased evaluations, including penetration testing, to identify vulnerabilities that internal teams might overlook. This provides an objective perspective on security. \\
    \cline{2-3}
    & Penetration Testing & Conducting regular penetration tests to simulate real-world attacks and identify vulnerabilities, assessing how well the organisation can handle potential attacks. \\
    \hline\hline

    \textbf{Response} & Incident Response Plans & Developing comprehensive, regularly updated incident response plans for various potential security incidents, ensuring preparedness and effectiveness through drills and simulations. \\
    \cline{2-3}
    & Immediate Action & Taking swift and decisive action during a security breach to limit damage. Clear, actionable plans should guide responses to minimise impact. \\
    \cline{2-3}
    & Post-Incident Analysis & Conducting thorough post-incident analyses to understand vulnerabilities, learn from breaches, and strengthen security measures to prevent future incidents. \\
    \cline{2-3}
    & Transparency & Communicating clearly with stakeholders, including customers, about security incidents to maintain trust while balancing the need to protect sensitive information. \\
    \hline\hline

    \textbf{Proactive} & Worst-Case Planning & Proactively planning for worst-case scenarios, conducting pre-emptive vulnerability assessments, and preparing for potential threats before they emerge. This approach helps prevent severe damage from cyber incidents. \\
    \cline{2-3}
    & Security by Design & Ensuring security is integrated into every phase of system development, addressing vulnerabilities early and continuously throughout the process, making security an integral part of the system. \\
    \cline{2-3}
    & Employee Training & Providing regular cybersecurity training for all employees to build security awareness and empower staff to be the first line of defence against internal threats and vulnerabilities. \\
    \cline{2-3}
    & Threat Hunting & Actively searching for potential vulnerabilities within the system before they can be exploited, strengthening defences by identifying risks that might not yet be apparent. \\
\end{longtable}