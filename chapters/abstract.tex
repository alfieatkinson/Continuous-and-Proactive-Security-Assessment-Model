\begin{abstract}
    This paper evaluates the Attention-Based View (ABV) theory in the context of information security risk management and proposes a new model called the Continuous and Proactive Security Assessment Model (CAPSAM). The ABV theory explains how Top Management Teams (TMT) allocate attention to cybersecurity issues, particularly in response to costly breaches. It suggests that TMTs have limited attention capacity and prioritise issues based on perceived importance. The theory is composed of three key principles: focus of attention, structural distribution of attention, and situated attention. While ABV effectively explains heightened TMT attention following a breach, it is limited by its reactive approach, focusing on responses after a breach and overlooking proactive security planning.

    This paper introduces CAPSAM to address the limitations of ABV and traditional risk assessment approaches. CAPSAM is a top-down model that promotes a proactive, continuous approach to security. The model is built on five pillars: Culture, Continuous Assessment, Auditing, Response, and Proactive Mitigation (CCARP). It operates within the FAMRM cycle (New Feature, Information Security Risk Assessment, Proactive Mitigation Strategies, Incident Response Planning, and Continuous Threat Monitoring), integrating security at every stage of system development.

    CAPSAM emphasises a security-first mindset, embedding security into agile development cycles and DevSecOps practices, with a strong focus on customer data protection. The model also highlights the importance of executive leadership in establishing a robust security posture, ensuring strategic alignment, and fostering organisational commitment to security. At the core of CAPSAM is its continuous improvement cycle, driven by ongoing risk assessments and feedback loops, which ensure that security remains adaptive in the face of emerging threats.

    While CAPSAM offers significant advantages, it also presents challenges, including the need for continuous updates to risk assessments and reliance on top management's commitment. Nevertheless, CAPSAM's proactive nature, focus on continuous improvement, and integration of security across an organisation make it a powerful framework for enhancing cybersecurity. By shifting from a reactive to a proactive approach, CAPSAM aims to reduce the likelihood of successful cyberattacks and enhance long-term resilience.
\end{abstract}