\begin{abstract}
    This paper evaluates the Attention-Based View (ABV) theory in the context of cybersecurity risk management and introduces the Continuous and Proactive Security Assessment Model (CAPSAM). The ABV theory suggests that a firm's behaviour is shaped by how decision-makers allocate their attention, with a focus on negative events such as high-cost cybersecurity breaches. While the ABV effectively explains why Top Management Teams (TMT) prioritise cybersecurity after a breach, it is limited by its reactive nature and does not address proactive security planning.
    
    CAPSAM, a ``top-down'' model, is designed to overcome these limitations by emphasising a proactive and continuous approach to risk assessment. The CAPSAM framework is built on five pillars: Culture, Continuous, Auditing, Response, and Proactive (CCARP), which integrate security considerations from the beginning of system development and throughout its lifecycle. The model operates within the proposed FAMRM cycle, ensuring that security is embedded in every stage of a system's development.
    
    The paper argues that a top-down approach, prioritising executive leadership and strategic integration, is essential for establishing a strong security posture. This approach addresses the limitations of bottom-up strategies by aligning security initiatives with organisational goals and fostering a ``security-first mindset''. CAPSAM is theoretically grounded in the ABV, Agile and DevSecOps principles, ISO 31000 risk management standards, organisational learning, and stakeholder/trust theories. Real-world cases of major data breaches reinforce the need for such a proactive approach.
    
    CAPSAM's strengths include its proactive nature, continuous improvement cycle, and focus on customer data protection. The model also poses challenges, including the need for continuous updates, resource allocation, and consistent application across large organisations. The TMT plays a huge role in CAPSAM's implementation by championing security as a strategic objective and ensuring proper resource allocation. Overall, CAPSAM offers a robust framework that addresses the shortcomings of reactive security models and provides a comprehensive approach to cybersecurity.
\end{abstract}    